\documentclass[oneside]{book}

\usepackage[russian]{babel}
\usepackage{enumitem}
\usepackage{amsmath}
\usepackage{amssymb}
\usepackage{tikz}
\usepackage{pgfplots}
\usepackage{amsthm}
\usepackage{empheq}
\usepackage{xcolor}
\usepackage{pgfplots}
\usepackage{graphicx}
\usepackage{wrapfig}
\usepackage[utf8]{inputenc}
\usepackage[T1, T2A]{fontenc}

\author{по конспектам лекций Рачковского Н.Н. \\ студентов группы 950501}

\newcommand{\boxedeq}[2]{\begin{empheq}[box={\fboxsep=6pt\fbox}]{align}\label{#1}#2\end{empheq}}
   \newtheorem{thm}{Тероэма}[chapter] % reset theorem numbering for each chapter

\pgfplotsset{compat=1.5}
\begin{document}
\title{Ответы на теоретические вопросы к экзамену по математике. \\ Семестр 1, 2019}
\maketitle

\chapter{Элементы теортии Множеств}
\section{Множества и операции над ними}

Множество - совокупность некоторых объектов, обладающих определёнными свойствами. Каждый из объектов
называется элементом обозначение множества: $\{a | P(a)\}$ где P(a) - свойство, объединяющее объекты a.\\

Специльные символы, обозначающие операции над множествами:

\begin{enumerate}
    \item содержится: A $\subseteq$ B. Каждый элемент множества А содержится в В.
    \item совпадает: $A = B \Leftrightarrow A \subseteq B, B \subseteq A$
    \item объединение: $A \cup B = \{c | c \in A \textbf{ или } c \in B\}$
    \item пересечение: $A \cap B = \{c | c \in A \textbf{ и } c \in B\}$
    \item теоритическо-множественная разность: $A \setminus B = \{c|c \in A \textbf{ и } c \notin B\}$
    \item декартово произведение: $A \times B = \{(a, b) | a \in A; b \in B\}$ \footnote{каждый элемент в паре с
          каждым другим, как при раскрытии скобок}
\end{enumerate}

Операции с $\emptyset$:

\begin{enumerate}
    \item $A \cup \emptyset = A$
    \item $A \cap \emptyset = \emptyset$
    \item $A \setminus \emptyset = A$
    \item $\emptyset \setminus A = \emptyset$
\end{enumerate}

% Рассматривая операции умножения и деления над \mathbb{N}

\section{Замкнутость множеств}

Рассматривая операции умножения и и деления над $\mathbb{N}$ мы \textit{остаёмся} в $\mathbb{N} \Rightarrow
\mathbb{N}$ замкнуто относительно операции умножения.Для того, чтобы $\mathbb{N}$ стало замкнуто относительно
операции вычитания нужно добавить к нему отрицательные числа и ноль тем самым привратив его в $\mathbb{Z}$.
Таким образом $\mathbb{Z}$ замкнуто относительно $\times, \pm$ но не $\div$. Для того, чтобы замкнуть
$\mathbb{Z}$ относительно $\div$, нужно дополнить его дробями вида $\frac{m}{n}$, где $m \in \mathbb{Z}$
и $n \in \mathbb{N}$. Т. О. получили $\mathbb{Q}$ Получили: $\mathbb{N} \subset \mathbb{Z} \subset \mathbb{Q}
\subset \mathbb{R}$ где $\mathbb{R}$ - действительные числа.

\section{Ограниченность множеств}

A ограничено сверху, если $\exists M, \forall a \in A : a \leq M$ и A ограничено снизу, если $\exists M, \forall a \in A : a \geq M$
\begin{quote}
    Таким образом, если множество ограничено \textbf{и} сверху \textbf{и} снизу, оно называется \textit{ограниченным}.
    $\Rightarrow \exists M, \forall a \in A : |a| \leq M$ (1)
\end{quote}

\begin{center}
    $\exists M_1, M_2, \forall a \in A : M_1 \leq a \leq M_2$\\
    $M = max(|M_1|, |M_2|)$\\
    $M \geq |M_1| \geq M_2$\\
    $M \geq |M_1| \Rightarrow -M \leq -|M_1| \leq M_1 \Rightarrow$\\
    $\forall a \in A : -M \leq -M_1 \leq a \leq M_2 \leq M \rightarrow -M \leq a \leq M$
\end{center}
Следовательно из ограниченности А получается (1).

\section{Окрестности}

Рассмотрим $a \in \mathbb{R}$. Окрестностью а является отрезок (b; c), содержущюю а.
Рассмотрим $\epsilon > 0$. $\epsilon$-окрестностью а является отрезок $(a-\epsilon; a+\epsilon)$, содержущюю а.\\
$\mathcal{U}_\epsilon(a)$ есть отрезок длиной  $2\epsilon$, центром которого является а: \\
$\mathcal{U}_\epsilon(a) = \{x \in \mathbb{R} | |x-a|< \epsilon\}$\\
Оно бывает и проколото: т.е. из отрезка удалена точка а: $\dot{\mathcal{U}}_\epsilon(a) = \mathcal{U} \setminus \{a\}$

\chapter{Функции}
\begin{quote}
    обведи пж важные уравнения в коробку boxedeq\{eq:*\}\{...\}
\end{quote}

Пусть даны 2 непустых множества А и В. Отображением из А и В называется правило, согласно которому каждому элементу
множества А соответствует не более одного элемента В. Это обозначается $f : A \rightarrow B$
Областью определения f называется множество $D(f) = \{a \in A | \exists b = f(a)\}$\footnote{f - заданное нами правило}
Множеством значений f называется множество $E(f) = \{b \in B | \exists a \in A; b = f(a)\}^2$ Запись
b = f(a) обозначает, что $a \in A$ в отображениии f соответствует $b \in B$ тут b - образ, а а - прообраз. \\
Свойства биективного\footnote{взаимооднозначного} отображения $f: A \rightarrow B$:
\begin{enumerate}
    \item $D(f) = A$
    \item $E(f) = B$
    \item $\forall a_1, a_2 \in A, a_1 \neq a_2: f(a_1) \neq f(a_2)$
    \item обратное оторажение: $f^{-1}: B \rightarrow A; a = f^{-1}(b) \Leftrightarrow b = f(a)$
\end{enumerate}
График отображения $f A \rightarrow B = \{(a, b) | b = f(a)\} \subset A \times B$ Если А и В - числовые, то это функция тогда
график функции есть подмножество в декартовом квадрате\footnote{$\mathbb{R}^2 = \mathbb{R} \times \mathbb{R}$}.
Рассмотрим полскость с прямоугольной системой координат: элементам множества $\mathbb{R}^2$ можно поставить в соответствие
точки этой полскости, координаты которой в этой С.К. являются  эти элементы $\mathbb{R}^2$. Тогда график функции можно предстваить как
множество точек, причем ясно, что не каждое множество точек задает график функции. Множество точек задает график функции тогда и только
тогда, когда любая вертикальная прямая параллельная оси ординат пересекает множество данных не более одного раза. Функция может задаваться
\textit{аналитически, графичекси и неявно}. Неявный способ: Рассмотрим $F : \mathbb{R}^2 \rightarrow R$ и Рассмотрим $F(x;y) = 0$.
На Координатной плоскости рассмотрим множество решений этого уравнения: $\{(x;y) \in \mathbb{R}^2 | F(x;y) = 0\}$: если оказывается, что
это множество является графиком функции, функция задана нефвно унавнением F(x;y) = 0.

\section{Типовые функции, график функции}
Линейная функция:\\
Функция вида $y = kx + b; k, b \in \mathbb{R}$ имеет графиком невертикальную прямую при b = 0 график функции проходит через (0; 0).
К - угловой коеффициент равный тангенсу кгла наклона графика к Ox. Взаимное расположение двух прямых, заданных функциями
$y_1 = k_1x + b_1$ и $y_2 = k_2x + b_2$:
\begin{enumerate}
    \item совпаление прямых $\Leftrightarrow k_1 = k_2; b_1 = b_2$
    \item параллельность прямых $\Leftrightarrow k_1 = k_2 \textbf{ и } b_1 \neq b_2$
    \item пересечение прямых $\Leftrightarrow k_1 \neq k_2$
\end{enumerate}

доказательство свойства 2:
\begin{center}
    $\Rightarrow)$ Пусть прямые $y_1 = k_1x + b_1$ и $y_2 = k_2x + b_2$ параллельны. Следовательно у них не общих точек:\\
    $\begin{cases}
        y = k_1x + b_1 \\
        y = k_2x + b_2
    \end{cases}$ не имеет решений \\ $\Rightarrow x(k_1 - k_2) = b_2 - b_1$ не имеет решений\\
    Следовательно $x = \frac{b_2-b_1}{k_1-k_2} \notin \mathbb{R} \Rightarrow \begin{cases}
        k_1 = k_2 \\ b_1 \neq b_2
    \end{cases}$\\
    $\Leftarrow)$ Предположим, что $\begin{cases} k_1 = k_2 \\ b_1 \neq b_2 \end{cases}$ и проведем все эти действия в обратном порядке.
\end{center}

\subsection{Формула получения угла между двумя прямыми}

$\begin{cases}
    y = k_1x + b_1 \\
    y = k_2x + b_2
\end{cases}$

\begin{tikzpicture}
    \begin{axis}
        \addplot [
            domain=-10:10,
            samples=100,
            color=red,
        ]
        {2*x + 1};
        \addplot [
            domain=-10:10,
            samples=100,
            color=blue,
        ]
        {4*x - 8};
    \end{axis}
    \node[text width=6cm, anchor=west, right] at (8,3)
    {обозначим угол между красной и синей линиями за $\theta$, наклон линий соответственно $\phi_1$ и $\phi_2$
    $\theta = \phi_1 - \phi_2$ \\ $k_1 = \tan{\phi_1}$ \\
    $k_2 = \tan{\phi_2}$ \\
    $\theta = \tan{\phi_1} - \tan{\phi_2} \Rightarrow$ \boxedeq{eq:*}{\theta  = \frac{k_1-k_2}{1+k_1k_2}}};
\end{tikzpicture}

Таким образом 2 прямые взаимоперпендикулярны тогда и только тогда когда $k_1 = \frac{-1}{k_2}$

\subsection{Основные элементарные функции}
Степенная функция \\
\begin{tikzpicture}
    \begin{axis}
        \addplot{x^2};
    \end{axis}
    \node[text width=2cm, anchor=west, right] at (0, -1) {n = 2n};
\end{tikzpicture}
\begin{tikzpicture}
    \begin{axis}
        \addplot{x^3};
    \end{axis}
    \node[text width=2cm, anchor=west, right] at (0, -1) {n = 2n+1};
\end{tikzpicture}\\
\begin{tikzpicture}
    \begin{axis}
        \addplot{x^(1/8)};
    \end{axis}
    \node[text width=6cm, anchor=west, right] at (0, -1) {$x^\frac{1}{n}$ где н - четное};
\end{tikzpicture}
\begin{tikzpicture}
    \begin{axis}
        [xmin=-5, xmax=5,
        ymin=-5, ymax=5]
        \addplot{x^(1/6)};
    \end{axis}
    \node[text width=6cm, anchor=west, right] at (0, -1) {$x^\frac{1}{n}$ где н - нечетное};
\end{tikzpicture}
ДОДЕЛАЙС

\chapter{Окружность, Эллипс, Гипербола, Парабола}
Пусть Существует прямоугольная система координат Oxy; Пусть даны две точки $A(x_1; y_1), B(x_2; y_2)$;
Тогда расстояние между А и В вычисляется так: \boxedeq{eq:*}{|AB| = \sqrt{(x_2-x_1)^2+(y_2-y_1)^2}}

\section{Фигуры и канонические уравнения фигур}
Говорят, что уравнение на плоскости задет некоторую фигуру, если принадлежность M(x; y) этой фигуре равносильно
выполнению равенства f(x; y) = 0 для каждой точки этой фигуры.

\subsection{Окружность}
\textbf{Окружностью называется множество всех точек в плоскости, удаленных от данной фиксированной точки, называемой
центром окружности на одно и то же расстояние, называемое радиусом окружности.} \\
дана точа M(x; y) и окружность с центром О($x_0, r_0$). М $\in \omega(O, r) \Leftrightarrow |MO|=R \Leftrightarrow
|MO|^2 = r^2 \Leftrightarrow$ \boxedeq{eq:*}{(x-x_0)^2+(y-y_0)^2 = r^2}
Равенство 3.2 есть уравнение окружности т.к. оно равносильно принадлежности точки М к окружности.

\subsection{Эллипс}
Пусть на плоскости заданы 2 точки $F_1, F_2$, расстояние между которыми равно 2с; и пусть дано
некоторое число $a > c$. \textbf{Эллипсом называется множество всех точек ранной плоскости, длял которых
сумма расстояний от этой точки до точек $F_1$ и $F_2 = 2a$.} Точки F называются фокусами эллипса. Вывод:
\begin{center}
        Зададим на плоскости ПСК с $Ox = F_1F_2$; координаты точек F получаются: $F_1(-c; 0), F_2(c; 0)$ \\
        Возьмем произвольную точку $M(x; y) \Rightarrow (MF_1 + F_1F_2) = 2a \Rightarrow$ \\
        $\sqrt{(x+c)^2 + y^2} + \sqrt{(x-c)^2 + y^2} = 2a$ \\
        $\therefore (x+c)^2 + y^2 = 4a^2 - 4a\sqrt{(x-c)^2 + y^2} + (x-c)^2 + y^2$ \\
        $\therefore a^2(x-c)^2+a^2y^2 = a^4 - 2a^2cx+c^2x^2$ \\
        \dots \\
        $\therefore b^2 = a^2 - c^2$ \\
        $\therefore b^2x^2 + a^2y^2 = a^2b^2$, делим на $a^2b^2$ \\
        \boxedeq{eq:*}{\frac{(x-x_0)^2}{a^2}+\frac{(y-y_0)^2}{b^2} = 1}\footnote{неуверен в записи, особенно в $(MF_1 + F_1F_2) = 2a$}
\end{center}
Так как обе переменных x и y в четных степенях, эллипс симметричен относительно начала координат.
Эллипс ограничен прямоугольником 2а на 2b. В случае совпадения a и b получим $\omega(0, a)$.
\textbf{эксцентриситет эллипса: $\varepsilon = \frac{c}{a}$}. $\varepsilon \in [0; 1] \therefore \varepsilon = 0$ для окружности.

\subsection{Гипербола}
На плоскости заданы несовпадающие точки $F_1, F_2$, расстояние между которыми равно 2с. Пусть $a \in (0; c)$.
\textbf{Гиперболой называется множество точек, для которых разность расстояний от точки до $F_1$ и $F_2$.} $F_1$ и $F_2$ это
фокусы гиперболы. На плоскости задана ПСК с $Ox = F_1F_2$; координаты точек F получаются: $F_1(-c; 0), F_2(c; 0)$
\begin{center}
    \pgfplotsset{every axis/.append style={
        axis x line=middle,    % put the x axis in the middle
        axis y line=middle,    % put the y axis in the middle
        axis line style={<->}, % arrows on the axis
        xlabel={$x$},          % default put x on x-axis
        ylabel={$y$},          % default put y on y-axis
        }}
    % arrows as stealth fighters
    \tikzset{>=stealth}
    \begin{tikzpicture}
    \begin{axis}[
    xmin=-5,xmax=5,
    ymin=-5,ymax=5]
    \addplot [red,thick,domain=-2:2] ({cosh(x)}, {sinh(x)});
    \addplot [red,thick,domain=-2:2] ({-cosh(x)}, {sinh(x)});
    \addplot[red,dashed] expression {x};
    \addplot[red,dashed] expression {-x};
    \draw[fill] (400,500) circle  [radius=1pt];
    \draw[fill] (600,500) circle  [radius=1pt];
    \end{axis}
    \end{tikzpicture}

    wywod urawnenija giperboly zdesja.

    \boxedeq{eq:*}{\frac{(x-x_0)^2}{a^2}+\frac{(y-y_0)^2}{b^2} = -1}

\end{center}
Так как обе переменных x и y в четных степенях, эллипс симметричен относительно начала координат.
$y = \pm\frac{b}{a}x$ - асимптоты гиперболы. a и b - полуоси гиперболыб точки пересеччения с Ox - вершины.
\textbf{эксцентриситет гиперболы: $\varepsilon = \frac{c}{a}$}. $c > a \Rightarrow \varepsilon >1$

\subsection{Парабола}
На плоскости задана прямая $\Delta$ и $F \notin \Delta$. \textbf{Параболой называется множество точек плоскости
равноудаленных от $\Delta$ и F.} При этом $\Delta$ - директрисса параболы, F - фокус Параболы. Введем ПСК:
Ox проходит через $F$ и $\perp \Delta \Rightarrow F(\frac{p}{2}; 0)$ где p - расстояние от F до $\Delta$.
\begin{center}
    Уравнение параболы

    wywod urawnenija tuta

    \boxedeq{eq:*}{y = \pm 2px}
\end{center}
y в уравнении в чтной степени $\Rightarrow$ парабола симметрична относительно Ox при $x \geq 0$ получается, \
что парабола расположена в правой полуплоскости.
\chapter{DPMW}
\pagebreak

\chapter{Числовая последовательность и ее предел. Свойства сходящихся последовательностей.}
Числовая последовательность называется отображением в котором каждому $\mathbb{N}$ числу соответствует
некоторое число. Последовательности принято изображать $\{x_n\} = x_1; x_2; \dots x_n$
Если из $\{x_n\}$ взято некое бесконечное подмножество, из которого сформирована другая последовательность,
в которой \textbf{порядок следования членов такой же как и в исходной последовательности, то она
называется подпоследовательностью.} Обозначение $\{{x_n}_m\}$.
Из определения последовательности: если $k_1 < k_2 \Rightarrow m_1 < m_2$.\\
Число а называется пределом последовательности \\ $\lim_{n \rightarrow \infty}{x_n = a} \Leftrightarrow$
$\forall\epsilon>0,  \exists N=N(\epsilon) \in \mathbb{N}, \forall n \geq N: |x_n - a| < \epsilon$
$\Rightarrow \lim_{n \rightarrow \infty}{x_n = a} \Leftrightarrow$ в сколь угодно малой $\mathcal{U}_\epsilon(a)$
может находиться \textbf{конечное число членов этой последовательности}.

Предел числовой последовательности есть точчка, в которой \textit{кучкуются} почти все члены последовательности
за исключением, может последнего члена.

Последовательность, имеющая предел называется \textit{сходящейся}; в противном случае - \textit{расходящейся}.
Расходящиеся последовстельности также включают бесконечно большие последовательности.
\begin{center}
    бесконечно большие последовательности:\\
    $lim_{n \rightarrow \infty}{k_n} = \infty \Leftrightarrow$ \\
    $\forall M > 0, \exists N=N(M) \in \mathbb{N}, \forall n \geq N: |x_n| > M$ \\
    бесконечно малые последовательности:\\
    $lim_{n \rightarrow \infty}{k_n} = -\infty \Leftrightarrow$\\
    $\forall M < 0, \exists N=N(M) \in \mathbb{N}, \forall n \geq N: |x_n| < M$ \\
\end{center}

\section{Свойства сходящихся последовательностей {\LARGE{DOKAZAT' SWOJSTWA}}}
\begin{enumerate}
    \item Сходящаяся последовательность имеет единственный предел. Действительно, если предположть,
          что пределов 2, можноуказать несколько $\mathcal{U}_\epsilon$ этих пределов, не пересекающте
          друг друга. По определению предела внутри каждой из этих  $\mathcal{U}_\epsilon(a)$ должно
          содержаться бесконечно много членов последовательности, что есть противоречие.
    \item Если Последовательность сходится к а, то любая подпоследовательность этоц последовательности сходиться к а.
    \item Любая мходящаяся последовательность ограничена: \begin{center}
        Пусть $\epsilon = 1: \exists \in \mathbb{N}, n \geq N: |x_n - a| < 1 \Leftrightarrow$
        $|x_n| - |a| \leq |x_n - a| < 1 \Leftrightarrow$ \\
        $|x_n| - |a| < 1 \Rightarrow |x_n| < |a| + 1$ \\
        Пусть члены $x_1 \dots x_{N-1}$, не попавшие в рассматриваемую окрестность точки а. и Пусть
        \textcolor{red}{$M = max(|x_1| \dots |x_{N-1}|, |a+1|)$} $\forall n, |x_n| \leq M$
        \end{center}
    \item Если для 2х членов последоватеьностей ${x_n}$ и ${y_n}$, сходящихся к числам a и b
          соответственно, начиная с некоторого номера $ x_n < y_n, a \leq b$: \begin{center}
                Пусть $\lim_{n \rightarrow \infty}{x_n = a}$ \\$\lim_{n \rightarrow \infty}{y_n = b}$ \\
                $a < b \Rightarrow \exists N \in \mathbb{N}, A_n \geq N: x_n < y_n$ \\
                Примем $\epsilon = \frac{b-a}{2}$ \\
                $\exists N_1, N_2 \in \mathbb{N}, \forall n \geq N_1, |x_n - a| < \frac{b-a}{2},$ \\
                                                  $\forall n \geq N_2, |y_n - b| < \frac{b-a}{2}$ \\
            $\therefore$ при $N = max(N_1, N_2)$ \\
            $\forall n \geq N: \begin{cases}
                \begin{cases}
                    x_n > a - \frac{b-a}{2} \\
                    x_n > a + \frac{b-a}{2}
                \end{cases} \\
                b - \frac{b-a}{2} < y_n < b + \frac{b-a}{2}
            \end{cases}$
          \end{center}
    \item Если для 3х последовательностей ${x_n}$, ${y_n}$, ${z_n}$ выполняется $x_n \leq y_n \leq z_n$
          $lim_{x_n \rightarrow \infty}{x_n = a}\; lim_{x_n \rightarrow \infty}{z_n = a}$, то ${y_n}$ также сходится к $a$
    \item Если $lim_{x_n \rightarrow \infty}{x_n = a \neq 0}$, то начиная с некоторого номера $|x_m| >
          \frac{a}{2}$ все члены этой последовательности имеют тот же знак, что и $a$.
    \item \begin{thm}
        Пусть ${x_n}$ и ${y_n}$ сходятся к $a$ и $b$, тогда \begin{enumerate}
            \item $\{x_n \pm y_n\} = k\; \lim_{n \rightarrow \infty}{k_n = a \pm b}$
            \item $\forall c \{c \cdot x_n\}\; \lim_{n \rightarrow \infty}{= c \cdot a}$
            \item $lim_{n \rightarrow \infty}{\{x_n \cdot y_n\} = a \cdot b}$
            \item $lim_{n \rightarrow \infty}\{{\frac{1}{\textcolor{red}{x_n}}\}{= \frac{1}{a}}}$, если $a \neq 0$
            \item $lim_{n \rightarrow \infty}{\{{\frac{y_n}{x_n}}\}= \frac{b}{a}}$, если $a \neq 0$
        \end{enumerate}
    \end{thm}
\end{enumerate}

\chapter{DPMW}
\pagebreak

\chapter{Монотонные последовательности, теорема Вейкерштрасса}
ебаьт где это в конспекте?

\chapter{DPMW}
\pagebreak

\chapter{Предел функции в точке и на бесконечности, Односторонние пределы.}

{\LARGE КАК-ТО МАЛО НАПИСАНО}

Предел функции на бесконечности определяется так:
\section{Бесконечный предел, Предел на бесконечности}
\begin{itemize}
    \item $\lim_{x \rightarrow \infty}{f(x)} = A \Leftrightarrow \\ \forall \epsilon > 0, \exists \delta > 0,
           \forall x, |x| > \delta; |f(x) - A| < \epsilon$
    \item $\lim_{x \rightarrow x_0}{f(x)} = \infty \Leftrightarrow \\ \forall \epsilon > 0, \exists \delta > 0,
           \forall x \in \dot{\mathcal{U}}_{\delta(x_0)}, |f(x)| > \epsilon$
\end{itemize}

\section{Односторонние пределы}
$y = f(x)$ определена на $(x-\delta; x)$. \\
$\lim_{x \rightarrow x_0 - 0}{f(x)} = A$: Односторонним пределом слева функции $y = f(x)$ называется $A:
\forall \epsilon > 0, \exists \delta_1 > 0, \forall x \in (x_0-\delta_0; x_0): |f(x)-A|<\epsilon$, если $A$ существует. \\
Анологично определяется предел справа: $\lim_{x \rightarrow x_0 + 0}{f(x)} = A$ $\forall \epsilon > 0, \exists \delta_1 > 0,
\forall x \in (x_0+\delta_0; x_0): |f(x)-A|<\epsilon$

\boxedeq{eq:*}{
    \lim_{x \rightarrow x_0}{f(x) = a} \Leftrightarrow \lim_{x \rightarrow x_0 - 0}{f(x)} = A = \lim_{x \rightarrow x_0 + 0}{f(x)}
}

\pgfplotsset{every axis/.append style={
        axis x line=middle,    % put the x axis in the middle
        axis y line=middle,    % put the y axis in the middle
        axis line style={<->}, % arrows on the axis
        xlabel={$x$},          % default put x on x-axis
        ylabel={$y$},          % default put y on y-axis
        title = предел слева(точка на красном) и справа(точка на синем)
        }}
    \begin{tikzpicture}
    \begin{axis}[
    xmin=-5,xmax=5,
    ymin=-5,ymax=5]
    \addplot [red,thick, domain=-4:2] expression {-3};
    \addplot [blue,thick, domain=2:5] expression {0.2*x + 3};
    \addplot [black, dashed] coordinates {(2,-3) (2, 3.1)};
    \draw (700,840) circle[radius=2pt];
    \draw (700,200) circle[radius=2pt];
    \end{axis}
    \node [text width=6cm, anchor=west, right] at (8,3) {в данном случае предела у функции нет};
    \end{tikzpicture}

\chapter{DPMW}
\pagebreak

\chapter{Непрерывность функций в точке, их свойства.}
$y = f(x)$ непрерывна в точке $x_0$, если она определена в этой точке, а также в $\mathcal{U}_{(x)}$ и при этом
$\lim_{x \rightarrow x_0}{f(x_0) \Leftrightarrow \\ \forall \epsilon > 0, \exists \delta > 0, \forall x, |x - x_0|
< \delta : |f(x) - f(x_0| < \epsilon }$
$\Delta_x = x-x_0$ - приращение аргумента \\
$\Delta f(x_0) = f(x) - f(x_0)$ - есть приращение функции в $x_0$ \\
$y = f(x)$ непрерывна в $x_0 \Leftrightarrow$
\boxedeq{eq:*}{
    \forall \epsilon > 0, \exists \delta > 0, |\Delta x| < \delta \Rightarrow |\Delta f(x_0)|
    < \epsilon \Leftrightarrow \lim_{\Delta x \rightarrow 0}{\Delta f(x_0)} = 0
}
\begin{quote}
    Непрерывность функции в точке означает то, что в любой, сколь угодно маленькой окрестности, бесконечно малое приращение аргумента
    влечёт за собой бесконечно маое приращение функции.
\end{quote}

Свойства непрерывной функции в точке
\begin{enumerate}
    \item Если функция непрерывна в точке $x_0$, тов некоторой окрестности этой точки эта функция ограничена.
    \item Если функция непрерывна в точке $x_0$ и $f(x_0) \neq 0$, то в некоторой окрестности $x_0$ функция имеет тот же знак, что и $f(x_0)$
    \item Если $y = f(x_0)$ и $y = g(x_0)$ непрерывна в точке $x_0$ и $f(x_0) < g(x_0)$, то $\exists \mathcal{U}_{(x_0)}$ где $f(x) < g(x)$
    \item Если $y = f(x_0)$ и $y = g(x_0)$ непрерывна в точке $x_0$, то так же непрерывны $y = f(x_0) \pm y = g(x_0)$, $y = f(x_0) \cdot y = g(x_0)$, $y = f(x_0) y \div g(x_0)$
    \item Непрерывность композиции функций: Если $y = g(x_0)$ непрерывна в точке $x_0$, $z = f(x_0)$ непрерывна в точке $y_0 = g(x_0)$, то
          $y = f(g(x))$ непрерывна в точке $x_0$.
          \begin{proof}
            $ \\
            \forall \epsilon > 0, \exists \delta > 0, \forall x \in \mathcal{U}_{\delta(x_0)}: |g(x) - g(x_0)|<\epsilon \\
            \forall \sigma > 0, \exists \tau > 0, \forall y \in \mathcal{U}_{\tau(y_0)}: |f(y) - f(y_0)|<\sigma \\
            \forall \sigma > 0, \exists \delta > 0, \forall x \in \mathcal{U}_{\delta(x_0)}: |f(g(x)) - f(g(x_0))|<\sigma \\
            $
            что и означает непрерывность $y = f(g(x))$ в точке $x_0$
        \end{proof}
\end{enumerate}

\section{Односторонняя непрерывность}

$y = f(x)$ определена на $(x_0 - \delta; x_0]$ такая функция называется непрерывной слева, если $\lim_{x \rightarrow x_0 - 0}{f(x)} = f(x_0)$
аналогично функция называется непрерывной справа, если $\lim_{x \rightarrow x_0 + 0}{f(x)} = f(x_0)$. Так как функция непрерывна,
она непрерывна слева и справа. \\
Функция называется разрывна в точке $x_0$, если она либо не определена в этой точке, либо определена, но не непрерывна. \\
Классификация точек разрыва:
\begin{enumerate}
    \item Если существуют и конечны оба односторонних пределаи эти односторонние пределы не равны друг другу, то эта точка - точка разрыва
          первого рода.
    \item Если функции справа равен пределу слева и не равен значению функции в точке, это точка устранимого разрыва.
          $lim_{x \rightarrow x_0+0}{f(x)} = lim_{x \rightarrow x_0-0}{f(x)} \neq f(x_0)$
    \item Если хотя бы один из односторонних пределов бесконечен или не существует - точка разрыва второго рода
\end{enumerate}

\pgfplotsset{every axis/.append style={
        axis x line=middle,    % put the x axis in the middle
        axis y line=middle,    % put the y axis in the middle
        axis line style={<->}, % arrows on the axis
        xlabel={$x$},          % default put x on x-axis
        ylabel={$y$},          % default put y on y-axis
        title = Точки разрыва
        }}
    \begin{tikzpicture}
    \begin{axis}[
    xmin=-5,xmax=5,
    ymin=-5,ymax=5]
    \addplot [red,thick, domain=-4:2] expression {-3};
    \addplot [blue,thick, domain=2:5] expression {0.2*x + 3};
    \addplot [black, dashed] coordinates {(2,-3) (2, 3.1)};
    \draw (700,840) circle[radius=2pt];
    \draw (700,200) circle[radius=2pt];
    \node[text width=6cm, anchor=west, right] at (0, 40) {первого рода};
    \end{axis}
    \end{tikzpicture}
    \begin{tikzpicture}
        \begin{axis}[
        xmin=-5,xmax=5,
        ymin=-5,ymax=5]
        \addplot [blue,thick, domain=-4:1.9] expression {0.2*x + 3};
        \addplot [blue,thick, domain=2.1:5] expression {0.2*x + 3};
        \addplot [black, dashed] coordinates {(2,1) (2, 3.1)};
        \draw (700,840) circle[radius=2pt];
        \draw [fill] (700,600) circle[radius=2pt];
        \node[text width=6cm, anchor=west, right] at (600, 60) {устранимого \\ разрыва};
        \end{axis}
        \end{tikzpicture} \\
        \begin{tikzpicture}
            \begin{axis}[
            xmin=-5,xmax=5,
            ymin=-5,ymax=5]
            \addplot [blue,thick, domain=-4:1.9] expression {0.2*x + 3};
            \addplot [blue,thick, domain=2.1:5] expression {1/(x-2)};
            \addplot [black, dashed] coordinates {(2,-5) (2, 5)};
            \node[text width=6cm, anchor=west, right] at (0, 20) {второго рода};
            \end{axis}
            \end{tikzpicture}
    \section{Непрерывными а любой точке ОДЗ являются}
    \begin{itemize}
        \item постоянные функции
        \item $y = x$
        \item $y = a_n x^m + a_{n-1} x^{m-1} + \dots + a_0$
        \item дробно-рациональные функции $y = \frac{P(x)}{Q(x)}$, P(x), Q(x) - многочлены степени х
        \item функции $\sin, \cos, \tan, \cot$
    \end{itemize}

    \chapter{DPMW}
    \pagebreak

    \chapter{Сравение функций, эквивалентные функции}
    Пусть $y = f(x)$ и $y = g(x)$ определены в $\mathcal{U}_{x_0}$. Говорят, что $f(x)$ сравнима с $g(x)$, если
    \boxedeq{eq:*}{
        \exists \epsilon, \exists \mathcal{U}_{x_0}, \forall x_0 \in \mathcal{U}_{x_0} : |f(x)| \leq \epsilon |g(x)|
    }
    В этом случае пишут, что $f(x) = O(g(x))$. \\
    Очевидно, что $f(x) = O(g(x))$ при $x \rightarrow x_0 \Leftrightarrow \lim_{x \rightarrow x_0}{\frac{f(x)}{f(x)}} \leq \epsilon$
    а это означает, что $\frac{f(x)}{f(x)}$ ограничена в $\mathcal{U}_{x_0}$.

    Говорят, что $y = f(x)$ бесконечно мала по сравнению $y = g(x)$ при $x \rightarrow x_0$, если
    $\forall \epsilon > 0, \exists \delta > 0, \forall x \in \mathcal{U}_{x_0} : |f(x)| < \\
    \textcolor{red}{\LARGE{HILFE\_MIR!\_ICH\_HABE\_DAS\_KONSPEKT\_NICHT!}}$ тогда пишут, что
    $f(x) = o(f(x))$ при $x \rightarrow x_0 \Rightarrow \lim_{x \rightarrow x_0}{|\frac{f(x)}{f(x)}|} = 0 \Leftrightarrow f(x0 = f(x) \cdot \alpha(x))$
    где $\alpha(x)$ - БМФ при $x \rightarrow x_0$.

    \section{Эквивалентность}

    Функции $y = f(x)$ и $y = g(x)$ квивалентны при $x \rightarrow x_0$, если $\lim_{x \rightarrow x_0}{\frac{f(x)}{g(x)}} = 1$ или
    конечному числу А, тогда пишется $f(x) \thicksim g(x)$ при $x \rightarrow x_0 \Rightarrow f(x) \thicksim g(x) \Leftrightarrow
    f(x) = g(x) + o(g(x))$, тут $y = g(x)$ - главная часть $y = f(x)$
    \begin{thm}
        Если $f(x) \thicksim g(x)$ при $x \rightarrow x_0$, то $\forall x$: \begin{itemize}
            \item $\lim_{x \rightarrow x_0}{f(x) \cdot h(x)} = lim_{x \rightarrow x_0}{g(x) \cdot h(x)}$
            \item $\lim_{x \rightarrow x_0}{\frac{f(x)}{h(x)}} = \lim_{x \rightarrow x_0}{\frac{g(x)}{h(x)}}$
        \end{itemize}
    \end{thm}

    Таблица эквивалентных при $x \rightarrow x_0$: \\
    \begin{center}
        \begin{tabular}{c|c}
            sin(x) & x\\
            tg(x) & x\\
            arcsin(x) & x\\
            arctg(x) & x\\
            $1 - cos(x)$ & $\frac{x^2}{2}$ \\
            $\ln{a}$ & x \\
            $a^x - 1$ & $x \cdot \ln{a}$ \\
            $\log_{a}{1+x}$ & $\frac{x}{\ln{a}}$ \\
            $e^x - 1$ & x \\
            $(1+x)^\beta - 1$ & $\beta x$ \\
            $x^\beta - 1$ & $\beta(x-1)$
        \end{tabular}
    \end{center}

\chapter{DPMW}

\chapter{Непрерывность функции на отрезке}
Пусть $y = f(x), [a;b] \subset \mathcal{D}(y)$. $y = f(x)$ непрерывна на $[a;b]$, если она непрерывна в каждой точке интервала $(a;b)$ и
непрерывна справа в точке $a$ и слува в точке $b$.
\begin{thm}
    Кантора о вложенных отрезках. \\
    Имеется $[a;b]$ и совокупность вложенных отрезков $[a;b] \supset [a_1;b_1] \supset [a_2;b_2] \supset \dots \supset [a_n; b_n]
    \supset \dots$ и при этом $\lim_{n \rightarrow \infty}{b_n - a_n} = 0$\footnote{вложены друг в друга и уменьшаются}, тогда
    \boxedeq{eq:*}{
        \exists a \in [a; b] : \lim_{n \rightarrow \infty}{a_n} = \lim_{n \rightarrow \infty}{b_n}
    }
\end{thm}
Используя теорему Кантора Докажем теорему Больцана-Вейерштрасса
\begin{proof}
    $\forall \{x_n\} \subset [a; b]$ можно выделить мходящуюся подпоследовательность:\\
    Разобьём $[a; b]$ точкой С пополам и рассмотрим $[a_1; b_1]$, половину первоначального отрезка.\\
    Эта половна содержит бесконечно много точек из $\{x_n\}$. Пусть $x_{n_1} \in [a_1; b_1]$. \\
    Точкой $C_2$ Разобьём отрезок $[a_1; b_1]$ пополам и мрассмотрим $[a_2; b_2]$, она содержит бесконечно много точек из $\{x_n\}$ \\
    и в этом отрезке обозначим $x_{n_k}$, чтобы $n_2 > n_1$ и так далее. Получим \\
    \begin{center}
        $\{x_{n_k}\} \in [a_k; b_k], \forall k \in \mathbb{N} \Rightarrow$ \\
        $a_k \leq x_{n_k} \leq, b_k - a_k = \frac{b_k - a_k}{2^k}$ \\
        $\lim_{n \rightarrow \infty}{\frac{b_k - a_k}{2^k} = 0}$ \\
        По теореме Кантора имеем: $\lim_{n \rightarrow \infty}{a_k} = \lim_{n \rightarrow \infty}{b_k} = a$\\
        В неравенстве $a_k \leq x \leq b_k$ перейдём к пределам.\\
        По теореме о 2х милиционерах:\\
        $a_0 \leq \lim_{n \rightarrow \infty}{x_{n_k}} \leq a_0 \Rightarrow \lim_{n \rightarrow \infty}{x_{n_k}} = a_0 \in [a; b]$
    \end{center}
\end{proof}
\begin{thm}
Если $y = f(x)$ непрерывна на $[a; b]$, то она ограничена на этом отрезке. \\
$ \exists c > 0, \forall x \in [a;b]: |f(x)| \leq c $
\end{thm}
\begin{proof}
    Пусть $y = f(x)$ непрерывна на $[a; b]$. Предположим, что она неограничена на этом отрезке. \\
    Отсюда $\forall n \in \mathbb{N}, \exists x_n \in [a;b]: |f(x)| \geq n$\\
    Отсюда по Больцана-Вейерштрасса в $\{x_n\}$ можно выделить сходящуюся подпоследовательность $\{x_{n_k}\}$ с пределом $x_0 \in [a; b]$\\
    Отсюда $\forall k, |f(x_{x_k})| > n_k, \lim_{k \rightarrow \infty}{|f(x_{x_k})|} \geq \infty$ \\
    Поскольку $\{x_n\} \rightarrow x_0$, в $x_0$ функция не является непрерывной, а терпит разрыв второго рода, что протеворечит нашему утверждению.
\end{proof}

\begin{thm}
    Вейерштрасса. \\
    Непрерывная на $[a; b]$ функция достинает на нём своего максимального и минимального значений.
\end{thm}

\chapter{DPMW}

\chapter{Производная функции, односторонние производные}

Пусть $y = f(x), x_0 \in \mathcal{D}(f(x))$. Рассмотрим график функции. и  прямые $y = k(x - x_0) + f(x_0)$
Среди всех таких прямвх рассмотрим ту, которая наиболее тесно прижимается к графику функции $f(x)$.
Такая прямая называется касательной к графику функции в точке $(x_0; f(x_0))$. Эту прямую можно найти так:
\textcolor{red}{На графике функции рассмотрим кроме $(x_0; f(x_0))$ рассмотрим $(x_1; f(x_1))$ и прямую, проходящую через эти точки. Эта прямая - секущая, приближённая\footnote{Размытое определение}}

\par Уравнение секущей с угловым коеффициентом. Так как секущая должна роходить через $(x_0; f(x_0))$ должно выпоняться равенство $k = \frac{f(x_1) - f(x_0)}{x_1 - x_0} \Rightarrow (x_1; f(x_1)) \rightarrow (x_0; f(x_0)) \Leftrightarrow x_1 - x_0 \Rightarrow
k = \lim_{x \rightarrow x_0}{\frac{f(x_1) - f(x_0)}{x_1 - x_0}}$ Если этот преел конечен и существует, то он есть производная функции
$y = f(x)$ в $x_0$ и обозначается $f'(x_0)$ \\
$x_1 - x_0 = \Delta x, f(x_1) - f(x_0) = \Delta f(x_0) \\ f'(x_0) = lim_{\Delta x \rightarrow 0}{\frac{\Delta f(x_0)}{\Delta x}}$ иногда обозначается $\frac{df(x_0)}{dx}$ \par Может оказаться, что $lim_{\Delta x \rightarrow 0}{\frac{\Delta f(x_0)}{\Delta x}}$ бесконечен,
в этом случае касательая к графику в точке вертикальна
\par Как известно, существование конечного предела равносильно существованию и равенству между собой односторонних пределов $lim_{\Delta x \rightarrow 0+0}{\frac{\Delta f(x_0)}{\Delta x}}$ и $lim_{\Delta x \rightarrow 0-0}{\frac{\Delta f(x_0)}{\Delta x}}$
Эти односторонние пределы, если они конечны и существуют, называются односторонними производными и обозначаются $f'(x_{0-0})$ и $f'(x_{0+0})$
Их существование означает существование касательной к фрагменту графика функции левее и правее $(x_0; f(x_0))$. Справедливо и обратное.
\par Возможны случаи, когда односторонние пределы существуют, но не равны друг другу это значит, что в точке $(x_0; f(x_0))$ терпит излом и не является гладким.

\pgfplotsset{every axis/.append style={
        axis x line=middle,    % put the x axis in the middle
        axis y line=middle,    % put the y axis in the middle
        axis line style={<->}, % arrows on the axis
        xlabel={$x$},          % default put x on x-axis
        ylabel={$y$},          % default put y on y-axis
        title = Излом графика функции
        }}
    \begin{tikzpicture}
    	\begin{axis}[
    		xmin=-5,xmax=5,
    		ymin=-5,ymax=5]
    		\addplot [red,thick, domain=-4:1] expression {-0.3*x + 1.2};
			\addplot [red,thick, domain=1:4] expression {-2.6*x + 3.5};
			\addplot [blue,thick, domain=-4:1] expression {sin(deg(x+1))};
			\addplot [blue,thick, domain=1:4] expression {-tan(deg(x-2)) - 0.6};
    	\end{axis}
    \end{tikzpicture}
\begin{thm}
	Если $f(x)$ имеет конечную производную в точке $x_0$, то она непрерывна в этой точке.
  \begin{proof}
    \begin{center}
      Пусть Существует конечный предел \\
      $\lim_{\Delta x \rightarrow 0}{\frac{\Delta f(x_0)}{\Delta x }} = f'(x_0) \Leftrightarrow \Delta f(x_0) = f'(x_0)+o(\Delta x)$ \\
      Перейдём к пределу при $\Delta x \rightarrow 0$: \\
      $\lim_{\Delta x \rightarrow 0}{\Delta f(x_0)} = 0 \Leftrightarrow f(x) \Leftrightarrow f(x_0)$ непрерывна в $x_0$ \\
    \end{center}
    Заметим, что обратное утверждение неверно.
  \end{proof}
\end {thm}
Так как производная - предел, из свойств пределов можно вывести свойства производных:
\begin{enumerate}
  \item $(f \pm g)' = f' \pm g'$
  \item $(cf)' = c(f)'$
  \item $(f \cdot g)' = f'g \cdot g'f$
  \item\footnote{proofs are pending} $(\frac{f}{g})' = \frac{f'g - g'f}{g^2}$
  \item $c' = 0$
\end{enumerate}
%\begin{table}
  \begin{center}
      \begin{tabular}{c|c}
        $f(x)$ & $f'(x)$ \\
        \hline \\
        $tg(x)$ & $\frac{1}{cos^2(x)}$ \\
          \hline \\
        $ctg(x)$ & $\frac{-1}{cos^2(x)}$ \\
          \hline \\
        $x^k$ & $k \cdot x^{x-1}$ \\
          \hline \\
        $e^x$ & $e^x$ \\
          \hline \\
        $log_a x$ & $\frac{1}{x \cdot ln(a)}$ \\
          \hline \\
        $ln(x)$ & $\frac{1}{x}$ \\
          \hline \\
        $arcsin(x)$ & $\frac{1}{\sqrt{1 - x^2}}$ \\
          \hline \\
        $arccos(x)$ & $\frac{-1}{\sqrt{1 - x^2}}$ \\
          \hline \\
        $arctg(x)$ & $\frac{1}{1 + x^2}$ \\
          \hline \\
        $arcctg(x)$ & $\frac{-1}{1 + x^2}$ \\
      \end{tabular}
  \end{center}
%\end{table}
Производная сложной функции:
\begin{itemize}
  \item $(f(g(x)))' = f'(g(x)) \cdot g'(x)$
  \item $(f^{-1}(y))' = \frac{1}{f'(x)}$ при $y = f(x)$
  \item $f'(x) = \frac{1}{f^{-1}(y)}$ при $y = f(x)$
\end{itemize}

\chapter{DPMW}
\pagebreak

\chapter{Основные правила дифференцирования, производные элементарных функций.}

Функция называется дифференцируемой в точке $x_0$, если её $\Delta f(\Delta x)$ можно предстваить так:
$f(x) - f(x_0) = A(x - x_0) + o(x - x_0)$ где $A$ - конечное число; $A(x - x_0)$ называется дифференциалом.
\begin{thm}
    Функция $y = f(x)$ дифференцируема в точке $x_0$ тогда и только тогда, когда функция имеет конечную производную в этой точке
    и производная функции равна $A$
    \begin{proof}
      Если $y = f(x)$ дифференцируема в $x_0$, то
      $$f(x) - f(x_0) = A(x-x_0) + o(x - x_0)\vert_{\div (x - x_0)}$$ \\
      при перезоде к пределам:
      $$\lim_{x \rightarrow x_0}{\frac{f(x) - f(x_0)}{x - x_0}} = \lim_{x \rightarrow x_0}{\frac{A+o(x-x_0)}{x-x_0}} = \large{A} \Rightarrow f'(x_0) = A$$
		Предположим, что $f(x)$ имеет конечную производную $$ \Rightarrow \lim_{x \rightarrow x_0}{\frac{f(x) - f(x_0)}{x - x_0}} =
		f'(x_0)$$ \\
		$$\frac{f(x) - f(x_0)}{x - x_0} = f'{x_0} + o(x-x_0)$$ \\ $$
		f(x) = f'(x_0)(x - x_0) + o(x-x_0) \cdot (x-x_0) \Rightarrow A = f'(x_0)$$
    \end{proof}
\end{thm}
Таким образом дифференцируемость функции равносильна существованию её конечной производной.
\boxedeq{eq:*}{ f(x) - f(x_0) = df(x_0) + (x - x_0)}
При $x \rightarrow x_0, df(x_0) = f'(x_0)(x - x_0)$ \\
Бесконечно малое приращение аргумента $\Delta x$ обозначается $dx$, отсюда
\boxedeq{eq:*}{df(x_0) = f'(x_0)dx} \\

Заметим, что формула справедлива и когда $x$ - функция.
\boxedeq{eq:*}{df(x(t)) = (f'(x(t)))' dt = f'(x) \cdot x(t)dt = f'(x)dx}

Дифференциал сожно использовать и при приблиэённом вычислении значения функции:\\
$$f(x) - f(x_0) = df(x_0) + o(x - x_0), x \rightarrow x_0 \Rightarrow$$ при $x$ близких к $x_0$
$o(x- x_0) \approx 0 \Rightarrow f(x) - f(x_0) \approx df(x_0) \Rightarrow$ \\
\boxedeq{eq:*}{f(x) \approx f(x_0) + df(x_0)}
\begin{center}
  Пример:
\end{center}

$$
  \sqrt[100]{1.1} \approx \vert_{x_0 \approx 1 = \sqrt{x}}_{\vert_{x = 1}} $$ \\
$$
  (1.1 - 1) + \sqrt[100]{1} = (x^{\frac{1}{100}})\vert_{x = 1} \cdot 0.1 + 1 = \frac{1}{100} \cdot x^{-0.99} \vert_{x = 1} \Rightarrow$$\\
$$
  0.1 \cdot \frac{1}{100}+1 = 1.001$$

\section{Основные свойства производной на отрезке}

\begin{thm}
  Ферма: Пусть $y = f(x)$ в точке $x_0$ имеет локальный экстремум\footnote{max || min} $\Rightarrow$ если
  $$\exists \mathcal{U}_{(x_0)} \forall x \in \mathcal{U}_{(x_0)}: f(x_0) \leq f(x) $$ \\ для мин. экстр $f(x_0) \geq f(x)}$
  \begin{proof}
    Если $x_0$ - точка локального максимума функции $f(x)$, то $\exists \mathcal{U}_{(x_0)} \forall x \in \mathcal{U}_{(x_0)}: f(x_0) \leq f(x) $.
    Рассмотрим односторонние пределы:
    $$
      \lim_{x \rightarrow x_0 - 0}{\frac{f(x) - f(x_0)}{x - x_0}} \geq 0, f(x) - f(x_0) \leq 0, x < x_0
    $$ \\ $$
      \lim_{x \rightarrow x_0 + 0}{\frac{f(x) - f(x_0)}{x - x_0}} \leq 0, f(x) - f(x_0) \leq 0, \frac{f(x) - f(x_0)}{x - x_0} \geq 0
    $$
    Так как функция дифференцируема в точке, то Существует предел, равный производной функции, равный обоим односторонним пределам и
    $$
    \begin{cases}
      f'(x_0) \geq 0 \\
      f'(x_0) \leq 0
    \end{cases} \Rightarrow f'(x_0) = 0
    $$
  \end{proof}
\end{thm}

\begin{thm}

Ролля: Пусть $y = f(x)$ непрерывна на $[a;b]$ и дифференцируема на $(a;b)$ и Если $f(a) = f(b), \exists c \in [a;b]: f'(c) = 0 \forall (a;b)$

\begin{proof}
  Если $f(x)$ не постоянна, то по теореме Вейерштрасса она достигает на этом отрезке своего маесимального и минимального значений, что не равны друг другу,
  а значит, чтчо хоть один их нах отличается от  $f(a) = f(b)$. Обозначим такую точку экстремума $c \in (a;b)$ \\
  $f(c) \neq f(a) = f(b)$ и по теореме Ферма $f'(c) = 0$
      \begin{center}
        \pgfplotsset{every axis/.append style={
                axis x line=middle,    % put the x axis in the middle
                axis y line=middle,    % put the y axis in the middle
                axis line style={<->}, % arrows on the axis
                xlabel={$x$},          % default put x on x-axis
                ylabel={$y$},          % default put y on y-axis
                title = удовлетв. усл.
                }}
            \begin{tikzpicture}
            	\begin{axis}[
            		xmin=-5,xmax=5,
            		ymin=-5,ymax=5]
            	   \addplot [red,thick, domain=-3:3] expression {cos(deg(3*x))+2};
                 \draw [fill] (800,500) circle[radius=2pt];
                 \node[text width=1cm, anchor=west, right] at (760, 450) {$b$};
                 \draw [fill] (200,500) circle[radius=2pt];
                 \node[text width=1cm, anchor=west, right] at (160, 450) {$a$};
                 \addplot [blue,thick, domain=-2.5:-1.75] expression {3};
                 \addplot [blue,thick, domain=-1.5:-0.75] expression {1};
                 \addplot [blue,thick, domain=-0.25:0.25] expression {3};
                 \addplot [blue,thick, domain=0.75:1.25] expression {1};
                 \addplot [blue,thick, domain=1.75:2.5] expression {3};
            	\end{axis}
            \end{tikzpicture}\\
            Для функции удовлетворяющей условиям теоремы Ролля обязательно найдётся точка на графике, касательной в которой будет горизонтальная прямая
          \end{center}
  \end{proof}
\end{thm}
\begin{thm}
  Каши: Пусть $y = f(x)$ и Пусть $y = g(x)$ непрерывны на $[a; b]$ и дифференцируемы на $(a;b), g'(x) \neq 0$, тогда $$
    \exists c \in (a;b): \frac{f(a) - f(b)}{g(a) - g(b)} = \frac{f'(c)}{g'(c)}
  $$
  \begin{proof}
    Пусть функция $F(x) = f(x) - f(a) - \frac{f(b) - f(a)}{g(b) - g(a)} \cdot (g(x) - g(a))$.
    Функция $F$ уодвлетворяет условиям теоремы Ролля $\Rightarrow \\ \exists c \in (a; b): F'(x) = 0$
    $$
      F'(x) = f'(x) - \frac{f(b)-f(a)}{g(b) - g(a)} \cdot g'(x)

      F'(c) = 0 \Leftrightarrow f(c) - \frac{f(b)-f(a)}{g(b) - g(a)} \cdot g'(c) = 0
    $$
  \end{proof}
\end{thm}

\chapter{Уравнение касательной и нормали к графику функции}
Данная глава находится в разработке, при отсутствии в ней полезной информации (или вообще какой-либо информации вините еврея, араба и немца (с явными расистскими наклонностями)
\pagebreak

\chapter{Основные правила дифференцирования. Производные элементарных функций}
Данная глава находится в разработке, при отсутствии в ней полезной информации (или вообще какой-либо информации вините еврея, араба и немца (с явными расистскими наклонностями)
\pagebreak

\chapter{Дифференциал функции}
Данная глава находится в разработке, при отсутствии в ней полезной информации (или вообще какой-либо информации вините еврея, араба и немца (с явными расистскими наклонностями)
\pagebreak

\chapter{Производные и дифференциалы высших порядков}
Данная глава находится в разработке, при отсутствии в ней полезной информации (или вообще какой-либо информации вините еврея, араба и немца (с явными расистскими наклонностями)
\pagebreak

\chapter{Дифференцирование функции, заданной параметрически}
Данная глава находится в разработке, при отсутствии в ней полезной информации (или вообще какой-либо информации вините еврея, араба и немца (с явными расистскими наклонностями)
\pagebreak

\chapter{Локальный экстремум функции, теорема Ферма(та, которая отстойная, а не которая великая терема Ферма)}
Данная глава находится в разработке, при отсутствии в ней полезной информации (или вообще какой-либо информации вините еврея, араба и немца (с явными расистскими наклонностями)
\pagebreak

\chapter{Теоремы Ролля, Лагранжа, Коши}
Данная глава находится в разработке, при отсутствии в ней полезной информации (или вообще какой-либо информации вините еврея, араба и немца (с явными расистскими наклонностями)
\pagebreak

\chapter{Правило Лопиталя}
Пусть функции $f(x)$ и $g(x)$ непрерывны и дифференцируемы в окрестности точки $x_{0}$ и обращаются в нуль в этой точке: $f(x_{0})=g(x_{0})=0$. Пусть $g'(x_{0})\neq 0$. Если существует предел $\lim_{x \to x_{0}} \dfrac{f'(x)}{g'(x)}$, то $\lim_{x \to x_{0}} \dfrac{f(x)}{g(x)} = \lim_{x \to x_{0}} \dfrac{f'(x)}{g'(x)}$.\\

\textbf{Замечание:}
Правило Лопиталя также справедливо, если  $\lim_{x \to x_{0}} f(x) =\lim_{x \to x_{0}} g(x) = \infty$\\

\textbf{Доказательство}\\
Функции $f(x)$ и $g(x)$ непрерывны и дифференцируемы в окрестности точки $x_{0}$, значит
$f(x_{0})=\lim_{x \to x_{0}} f(x) = 0$ и $g(x_{0})=\lim_{x \to x_{0}} g(x) = 0$. По теореме Коши для отрезка $[x_{0};x]$, лежащего в окрестностях $x_{0}$ существует $\dfrac{f(x)-f(x_{0})}{g(x)-g(x_{0})}=\dfrac{f'(c)}{g'(c)}$, где $c$ лежит между точками $x$ и $x_{0}$. Учитывая, что  $f(x_{0})=g(x_{0})=0$, получаем \begin{center}
	$\dfrac{f(x)}{g(x)}=\dfrac{f'(c)}{g'(c)}$.
\end{center} При $x\to x_{0}$ $c$ также стремится к $x_{0}$; перейдем к пределу: \begin{center}
	$\lim_{x \to x_{0}} \dfrac{f(x)}{g(x)} = \lim_{c \to x_{0}} \dfrac{f'(c)}{g'(c)}$.

\end{center}
Получается$\lim_{x \to x_{0}} \dfrac{f(x)}{g(x)} = \lim_{c \to x_{0}} \dfrac{f'(c)}{g'(c)}$, а $\lim_{x \to x_{0}} \dfrac{f'(x)}{g'(x)} = \lim_{c \to x_{0}} \dfrac{f'(c)}{g'(c)}$, значит \begin{center}
	\boxedeq{eq:*}{\lim_{x{\large } \to x_{0}} \dfrac{f(x)}{g(x)} = \lim_{x \to x_{0}} \dfrac{f'(x)}{g'(x)}}
\end{center}
А если кратенько, то полученную формулу можно читать так: \textbf{предел отношения двух
	бесконечно малых равен пределу отношения их производных, если по­
	следний существует.}\\

\textbf{Замечания:}
\begin{enumerate}

	\item Правило Лопиталя справедливо и в случае, когда функции $f(x)$ и $g(x)$ не определены при  $x=x_{0}$, но $\lim_{x \to x_{0}} f(x) = 0$ и $\lim_{x \to x_{0}} g(x) = 0$. В этом случае $f(x_{0})=\lim_{x \to x_{0}} f(x) = 0$ и $g(x_{0})=\lim_{x \to x_{0}} g(x) = 0$
	\item Правило Лопиталя справедливо и в случае, когда ${x \to \infty}$: \begin{center}
		$\lim_{x \to \infty} \dfrac{f(x)}{g(x)} = \lim_{x \to \infty} \dfrac{f'(x)}{g'(x)}$
	\end{center}
	\item Если производные $f'(x)$ и $g'(x)$ удовлетворяют тем же условиям, что и $f(x)$ и $g(x)$, то правило Лопиталя можно применить еще раз:\boxedeq{eq:*}{\lim_{x{\large } \to x_{0}} \dfrac{f(x)}{g(x)} = \lim_{x \to x_{0}} \dfrac{f'(x)}{g'(x)}=\lim_{x \to x_{0}} \dfrac{f''(x)}{g''(x)}}

\end{enumerate}

\textbf{Виды неопределенностей:}
\begin{enumerate}
	\item Неопределенность вида $\dfrac{0}{0}$:

	$\lim_{x \to 0} \dfrac{1-cos(6x)}{2x^{2}}=
	[\dfrac{0}{0}]=
	\lim_{x \to 0} \dfrac{(1-cos(6x))'}{(2x^{2})'}=
	\lim_{x \to 0} \dfrac{6sin(6x)}{4x}=
	\dfrac{3}{2}\lim_{x \to 0} \dfrac{sin(6x)}{x}=
	\dfrac{3}{2}\times[\dfrac{0}{0}]=
	\dfrac{3}{2}\lim_{x \to 0} \dfrac{(sin(6x))'}{(x)'}=
	\dfrac{3}{2}\lim_{x \to 0} \dfrac{6cos(6x)}{1}=
	\dfrac{3}{2}\times 6=9$

	\item Неопределенность вида $\dfrac{\infty}{\infty}$:

	$\lim_{x \to \frac{\pi}{2}} \dfrac{tg(3x)}{tg(5x)}=
	[\dfrac{\infty}{\infty}]=
	\lim_{x \to \frac{\pi}{2}} \dfrac{(tg(3x))'}{(tg(5x))'}=
	\lim_{x \to \frac{\pi}{2}}\dfrac{3cos^{2}(5x)}{5cos^{2}(3x)}=
	\dfrac{3}{5}\times[\dfrac{0}{0}]=
	\dfrac{3}{5}\lim_{x \to \frac{\pi}{2}}\dfrac{cos^{2}(5x)-1+1}{cos^{2}(3x)-1+1}=
	\dfrac{3}{5}\lim_{x \to\frac{\pi}{2}}\dfrac{cos(10x)+1}{cos(6x)+1}=
	\dfrac{3}{5}\times[\dfrac{0}{0}]=
	\dfrac{3}{5}\lim_{x \to \frac{\pi}{2}} \dfrac{(cos(10x)+1)'}{(cos(6x)+1)'}=
	\dfrac{3}{5}\lim_{x \to \frac{\pi}{2}} \dfrac{10sin(10x)}{6sin(6x)}=
	\lim_{x \to \frac{\pi}{2}} \dfrac{sin(10x)}{sin(6x)}=
	[\dfrac{0}{0}]=
	\lim_{x \to \frac{\pi}{2}} \dfrac{(sin(10x))'}{(sin(6x))'}=
	\lim_{x \to \frac{\pi}{2}} \dfrac{10cos(10x)}{6cos(6x)}=
	\dfrac{5}{3}$

	\textbf{Для пунктов 3-7 рассмотрим преобразования в общих случаях:}

	\item Неопределенность вида $\infty-\infty$:

	 Пусть $f(x)\to\infty, g(x)\to\infty$ при $x\to x_{0}$, тогда:\\

	 $\lim_{x \to x_{0}} (f(x)-g(x))=
	 [\infty-\infty]=
	 \lim_{x \to x_{0}} (\dfrac{1}{\dfrac{1}{f(x)}}-\dfrac{1}{\dfrac{1}{g(x)}})=
	 \lim_{x \to x_{0}} (\dfrac{{\dfrac{1}{g(x)}}-{\dfrac{1}{f(x)}}}{{\dfrac{1}{f(x)}}{\dfrac{1}{g(x)}}})=
	 [\dfrac{0}{0}]=...$

	\item Неопределенность вида $\infty\times 0$:

	Пусть $f(x)\to\ 0, g(x)\to\infty$ при $x\to x_{0}$, тогда:\\

	$\lim_{x \to x_{0}} (f(x)g(x))=
	[\infty\times 0]=
	\lim_{x \to x_{0}} \dfrac{f(x)}{\dfrac{1}{g(x)}}=
	\dfrac{0}{0}=...$

	\item Неопределенность вида $1^{\infty}$
	\item Неопределенность вида $\infty^{0}$
	\item Неопределенность вида $0^{0}$\\

	\textbf{Для неопределенностей вида 4-7 воспользуемся следующим преобразованием:}

	Пусть $f(x)\to\ 1, g(x)\to\infty$; или $f(x)\to\infty, g(x)\to 0$; или $f(x)\to\ 0, g(x)\to 0$ при $x\to x_{0}$. Для нахождения предела вида $\lim_{x\to x_{0}} f(x)^{g(x)}$ удобно сначала прологарифмировать выражение
	 \begin{center}
		$A=f(x)^{g(x)}$
	\end{center}

\end{document}

%% {}
